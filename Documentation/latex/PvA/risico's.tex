
% !TeX spellcheck = nl_NL
\section{Risico's}

De risico's zijn onder te verdelen in interne- en externe risico's. De interne risico's zijn risico's waar de projectgroep direct invloed op kan uitoefenen. De externe risco's zijn risico's zijn bedreigingen van buitenaf. Ieder risico wordt op twee fronten gewogen op een schaal van 1 tot 5. De impackt maal het gevolg bepaald de grote van het risico.  \\[0.5cm]

\newpage
\begin{landscape}
\subsection{Interne risico's}
\begin{table}[]
	\centering
	\caption{Interne risico's}
	\label{intrisico}
	\begin{tabular}{llllll}
		Nr. & Risico                                                                                & Omschrijving                                                                                                                                           & \begin{tabular}[c]{@{}l@{}}Kans op \\ optreden\end{tabular} & Impact & \begin{tabular}[c]{@{}l@{}}Grote \\ risico\end{tabular} \\
		1   & Niet de juiste UR robot                                                               & \begin{tabular}[c]{@{}l@{}}Door de beperkte beschibaarheid van de robots kan het mogelijk zijn\\ dat niet de juiste robot beschikbaar is.\end{tabular} & 3                                                           & 3      & 9                                                       \\
		2   & Geen juiste end of arm tool                                                           & De end of arm tool(EOAT) is het meest kritische onderdeel van de robot.                                                                                & 3                                                           & 5      & 15                                                      \\
		3   & \begin{tabular}[c]{@{}l@{}}Slechte communicatie naar de \\ opdrachtgever\end{tabular} & \begin{tabular}[c]{@{}l@{}}De opdrachtgever is niet goed op de hoogte van de voortgang en daardoor \\ niet tevrede met het resultaat\end{tabular}      & 2                                                           & 5      & 10                                                      \\
		4   & Beperkte kennis/Ervaring                                                              & \begin{tabular}[c]{@{}l@{}}De groepsleden bezitten niet over genoeg kennis om het probleem op te \\ lossen\end{tabular}                                & 2                                                           & 4      & 8                                                       \\
		5   & Onjuiste planning                                                                     & \begin{tabular}[c]{@{}l@{}}De planning is niet juist opgesteld waardoor de beschikbare tijd niet juist\\ wordt gebruikt\end{tabular}                   & 2                                                           & 3      & 6                                                      
	\end{tabular}
	\end{table}
\end{landscape}

\newpage

\textbf{Beheersmaatregelen}
\begin{itemize}
	\item [1] Wanneer de robot niet beschikbaar is wordt er in overleg met de opdrachtgever besloten hoe de opstelling kan worden aangepast zodat er met de Kawasaki robot kan worden gewerkt. 
	\item [2] Door meerdere prototypes te ontwerpen kan dit risico's worden verkleind. Daarnaast zal er door het aangeleerde ontwerpproces worden gelopen op tot goede concepten te komen.
	\item[3] Wekelijkse update sturen naar de opdrachtgever. Dit lijkt op de wekelijkse pecha kucha. Hierin wordt kort de behaalde activiteiten beschreven en eventueel foto- en film materiaal mee gestuurd.
	\item [4] De school bied door middel van lessen de kans om genoeg kennis op te doen. Door aanwezig te zijn bij deze lessen kan dit risico worden beperkt.
	\item [5] Aan het begin van iedere werkweek is er een korte bijeenkomst(scrum-meeting) met de projectleden waarin de planning wordt doorgesproken. Hierin wordt bepaald of het project op schema loopt en wat de knelpunten die week zijn.
	
\end{itemize}

\newpage
\begin{landscape}
\subsection{Externe risco's}
\begin{table}[]
	\centering
	\caption{Externe risico's}
	\label{exrisico}
	\begin{tabular}{llllll}
		Nr. & Risico                                  & Omschrijving                                                                                                                                         & \begin{tabular}[c]{@{}l@{}}Kans op \\ optreden\end{tabular} & Impact & \begin{tabular}[c]{@{}l@{}}Grote \\ risico\end{tabular} \\
		1   & Niet in contact komen met opdrachtgever & \begin{tabular}[c]{@{}l@{}}Door eventuele ziekte of vakantie kan de opdrachtgever \\ niet bereikbaar zijn\end{tabular}                               & 2                                                           & 4      & 8                                                       \\
		2   & Lange levertijden                       & \begin{tabular}[c]{@{}l@{}}Wanneer er speciale onderdelen nodig zijn is er kan dat \\ de levertijd lang is en dit het project vertraagt\end{tabular} & 2                                                           & 3      & 6                                                      
	\end{tabular}
\end{table}

\textbf{Beheersmaatregelen}
\begin{itemize}
	\item [1] In overleg met de opdrachtgever een tweede contractpersoon opstellen.
	\item [2] Door in een vroeg stadium een componenten lijst op te stellen kan worden bepaald of er onderdelen zijn met een kritische levertijd en hiernaar worden gehandeld.
\end{landscape}	
\newpage
